\documentclass[final]{fhnwreport}         %[mode] = draft or final
\input{fhnwheader}            %loads all packages, definitions and settings

%%%%% Logo: Hocvhschule HTU oder HSI, Sprache DE oder EN:
%\newcommand{\logofilename}{FHNW_HTU_EN}
\newcommand{\logofilename}{FHNW_HSI_DE}
%\newcommand{\logofilename}{FHNW_HSI_EN}
%%%%%
%%%%% Bibliographie entweder im IEEE- oder im APA-Stil:
\usepackage[style=ieee,urldate=comp,backend=biber]{biblatex}
%\usepackage[style=apa,urldate=comp,backend=biber]{biblatex}
%%%%%
\usepackage{xcolor}

\usepackage[automake]{glossaries}

\newglossaryentry{maths}
{
	name=mathematics,
	description={Mathematics is what mathematicians do}
}
\newacronym{abb:HTTP}{HTTP}{Hypertext Transfer Protocol}
\makeglossaries



\lstdefinestyle{customtypescript}{
  belowcaptionskip=1\baselineskip,
  breaklines=true,
  frame=single,
  xleftmargin=\parindent,
  numbers=left,
  language=Java, % absichtlich Java für bessere Farben
  showstringspaces=false,
  basicstyle=\footnotesize\ttfamily,
  keywordstyle=\color{black}, % Verhindert falsches Blau bei z. B. „label“
  commentstyle=\itshape\color{green!60!black},
  identifierstyle=\color{black},
  stringstyle=\color{orange!80!black},
  numberstyle=\tiny\color{gray!70},
  rulecolor=\color{gray!70},
  backgroundcolor=\color{gray!5},
  morekeywords={}, % verhindert zusätzliche Keywords
  literate={ß}{{\ss}}1
           {Ö}{{\"O}}1 {Ä}{{\"A}}1 {Ü}{{\"U}}1
           {ü}{{\"u}}1 {ä}{{\"a}}1 {ö}{{\"o}}1
           {`}{\textasciigrave}1
           {'}{{\textquotesingle}}1
           {"}{{\textquotedbl}}1
}


\addbibresource{literature/beispiel_bib.bib}
											
\title{Code Assist und Navigation für IoT-Konfigurationen
	mit Language Server Protocol}  %Project Title
\author{Bachelor Thesis}    %Document Type => Technical Report, ...
\date{Windisch, August 2025}               %Place and Date

\begin{document}

\pagenumbering{roman}	

%%---TITLEPAGE---------------------------------------------------------------------------
\selectlanguage{ngerman}                  %ngerman or english
\maketitle

\vfill

\begin{figure}[H]
\centering
%\includegraphics[width=\linewidth]{} % Pic Title
\end{figure}

\vfill

\begin{tabular}{@{}p{5cm} l}
Student          			&    Gianni Parrillo\\[2ex]        
Experte						&    Raphael Schweizer Iuliano\\[2ex]
Fachbetreuer				&    Prof. Dr. Dominik Gruntz\\
							&	 Daniel Kröni\\[2ex]
Auftraggeber				&    Fabrizio Parrillo, Colomba Link GmbH\\[3ex]
Projektnummer				&    $25FS\_IMVS39$\\[4ex]
\multicolumn{2}{@{}l}{Fachhochschule Nordwestschweiz, Hochschule für Informatik}
\end{tabular}

\vspace*{4ex}
% Beispiel für Logo Industriepartner
\begin{tikzpicture}[remember picture,overlay,every node/.style={anchor=north east}]
  \node at (current page.north east) [xshift=-1cm, yshift=-0.5cm] {\includegraphics[width=4cm]{colomba.png}};
  % Photo by Stoica Ionela on Unsplash
\end{tikzpicture}

\clearpage


%%---ABSTRACT----------------------------------------------------------------------------
\selectlanguage{ngerman}				%ngerman or english
\thispagestyle{empty}
\section*{Abstract}
Die Colomba ....

\vspace{2ex}

\textbf{Keywords:}

tic, tac

\clearpage

\section*{Vorwort / Dank}




%%---TABLE OF CONTENTS-------------------------------------------------------------------	
\selectlanguage{ngerman}				%ngerman or english
\tableofcontents
\clearpage
\listoffigures
\newpage
\addcontentsline{toc}{section}{Listingverzeichnis}
\renewcommand{\lstlistlistingname}{Listingverzeichnis}
\lstlistoflistings
\newpage
\addcontentsline{toc}{section}{Glossar}
\printglossaries

\newpage
%%---TEXT--------------------------------------------------------------------------------
\pagenumbering{arabic}

\section{Einleitung}
 In diesem Kapitel wird die Auftragsgeberfirma Colomba Link vorgestellt. Es wird die Ausgangs
lage beschrieben und das zu lösende Problem wird eingeführt. Es werden grundlegende Begriffe
 definiert, die in dieser Thesis von Bedeutung sind.

\subsection{Colomba Link}

Die Colomba Link GmbH entwickelt mit Monidas eine IoT-Plattform zur Vernetzung, Verwaltung und Auswertung industrieller Anlagendaten. Sie richtet sich an IoT-Spezialisten sowie Endnutzer und bietet Funktionen von der Einrichtung der Sensoren bis zur automatisierten Überwachung und Alarmierung. Derzeit konfigurieren technische Mitarbeitende die Sensoren kundenspezifisch über eine Weboberfläche (Monidas-Plattform). Komponenten wie Sensoren, Regeln und Benachrichtigungen müssen dabei einzeln erstellt und anschliessend manuell über mehrere Eingabemasken miteinander verknüpft werden. Mit zunehmender Anzahl von Sensoren steigt dadurch der Aufwand für das technische Personal. Aufgrund begrenzter Entwicklungsressourcen verfolgt das Startup das Ziel einer generischen Implementierung, welche Anpassungen und Erweiterungen des Datenmodells sowie den Verwaltungsprozess erleichtert und langfristig zu höherer Kundenzufriedenheit beiträgt. 

\subsection{Konfigurationsprozess}
Zum Verständnis der in dieser Arbeit beschriebenen Lösung ist es erforderlich, den Aufbau und die Konfiguration der Sensoren in der Monidas-Plattform grundlegend zu kennen. Der folgende Abschnitt erläutert den aktuellen Ablauf der Sensorkonfiguration und stellt relevante Begriffe vor. Dabei beschränkt sich die Darstellung auf jene Aspekte, die für die Problemstellung und die entwickelte Lösung von Bedeutung sind.

Die Konfiguration eines Sensors auf der Monidas-Plattform erfolgt über Eingabemasken. Dabei handelt es sich um Benutzeroberflächen, auf denen Informationen eingegeben und Optionen ausgewählt werden können. Erst wenn alle erforderlichen Konfigurationselemente erstellt wurden, können sie im Monitor zusammengeführt werden. Dieser verknüpft sämtliche Elemente miteinander und ermöglicht dadurch die Überwachung des Sensors.

Im Folgenden werden die einzelnen Konfigurationselemente näher beschrieben:

Actions definieren, wann und unter welchen Bedingungen eine Benachrichtigung versendet wird. Eine Benachrichtigung ist eine Mitteilung, die an vordefinierte Empfängergruppen gesendet wird. Dabei kann festgelegt werden, dass eine Benachrichtigung bei bestimmten Sensorzuständen (Idle, Alert oder Alarm) ausgelöst wird. Zusätzlich wird konfiguriert, ob die Benachrichtigung bei jedem Auftreten des Zustands oder nur einmalig versendet werden soll. Eine Benachrichtigung setzt sich aus den Konfigurationselementen Gruppe und Template zusammen.

Groups sind Vorlagen, in denen definiert ist, an welche Empfänger eine Benachrichtigung gesendet wird. Die entsprechenden E-Mail-Adressen werden innerhalb der Gruppe hinterlegt. Aktuell unterstützt die Plattform ausschließlich den Kommunikationskanal E-Mail.

Templates sind Vorlagen, in denen der Inhalt einer Benachrichtigung festgelegt wird. Sie bestehen aus einem Titel und einer Mitteilung. Der Titel wird beispielsweise als Betreff einer E-Mail verwendet, während die Mitteilung den eigentlichen Benachrichtigungstext enthält.

Die Konfigurationskomponente Function ist für die Logik der Sensordatenauswertung zuständig. Das technische Personal definiert diese Logik direkt im Editor der Monidas-Plattform mithilfe von JavaScript-Code. Dabei handelt es sich um ein einfaches Texteingabefeld ohne Entwicklungsunterstützung.

Zusätzlich kann technisches Personal ein JSON-Schema erstellen, in dem Variablen als Properties definiert werden. Innerhalb der programmierten Logik kann anschliessend auf diese Properties über die im Code verfügbare Variable ruleConfig zugegriffen werden. Die Zuweisung der Properties erfolgt entweder direkt innerhalb der Funktion oder zu einem späteren Zeitpunkt über die Benutzeroberfläche im Rahmen der Sensorkonfiguration. 

Dabei dient das General-Schema als Master-Schema, welches sämtliche definierten Variablen enthält und vom technischen Personal gepflegt wird. Basierend darauf lässt sich zusätzlich ein reduziertes User-Schema definieren. Dieses User-Schema bildet zunächst das General-Schema vollständig ab. Danach können einzelne Properties entfernt werden, indem sie einfach aus dem User-Schema gelöscht werden. Nur die im User-Schema verbliebenen Properties sind später in der Benutzeroberfläche für Endnutzer (z. B. Kunden) sichtbar.

Ein konkretes Beispiel für eine solche Funktion zeigt der Temperaturchecker in Abbildung \ref{fig:function}. Diese Funktion analysiert kontinuierlich die eingehenden Sensordaten und entscheidet basierend auf vordefinierten Grenzwerten, ob sich der Sensorzustand im Status ok, alert oder alarm befindet. Wird beispielsweise ein Grenzwert mehrfach überschritten, ändert sich der Zustand des Sensors entsprechend.

\begin{figure}[H]
  \centering
  \includegraphics[width=1\linewidth]{function.png}
  \caption{Konfigurationskomponente einer Temperatur Checker Funktion}
  \label{fig:function}
\end{figure}


(Playground wird evtl. kurz beschrieben TBD)

Die Hauptkomponenten des Monitors wird noch beschrieben(TBD) mit einer Abbildung. 



\subsection{Problemstellung}

Info: Die Problemstellung wurde noch nicht angepasst auf das neue Kapitel 1.2 (TBD).

Basierend auf den Erkenntnissen und Einschränkungen des Vorprojekts IP5 sollen in dieser Bachelorarbeit die hardcodierten Komponenten reduziert werden. Ziel ist die Analyse und Umsetzung einer modularen Lösung, welche das Datenmodell als virtuelles Filesystem bereitstellt, um Anpassungen oder Erweiterungen am Datenmodell ohne Änderungen am Quellcode zu ermöglichen.Die entwickelte Lösung soll mit verschiedenen Datenstrukturen kompatibel sein und insbesondere die Abbildung komplexer IoT-Strukturen unterstützen. Dabei wird das spezifische Monidas-Datenmodell zunächst durch ein allgemeineres Modell ausgetauscht. Zur Validierung und Überprüfung der angestrebten Generalisierbarkeit wird am Ende der Arbeit das ursprüngliche Monidas-Datenmodell exemplarisch wieder integriert. Zusätzlich wird ein eigener Language Server implementiert, welcher erweiterte Funktionen wie Validierung, Autovervollständigung und Navigation im Editor bereitstellt und somit die Bearbeitung der Daten  unterstützt.

Ein erster Schritt zur Reduktion des Entwicklungsaufwandes wurde im Vorprojekt IP5 („Entwicklung einer VS-Code Extension zur Verwaltung von IoT-Daten in der Monidas-Plattform“) evaluiert. Im Rahmen eines Proof-of-Concept untersuchte das Projekt, ob eine alternative Benutzeroberfläche den bisherigen Prozess der Sensorkonfiguration vereinfachen und optimieren kann. Dazu wurde eine VS-Code Extension entwickelt, die Teile des bestehenden Monidas-Datenmodells in einer hierarchischen Baumstruktur (TreeView) darstellt. Diese Benutzeroberfläche ermöglicht die direkte Erstellung und Bearbeitung der IoT-Daten in der Entwicklungsumgebung mithilfe eines JSON-Editors. Obwohl der entwickelte Prototyp aus IP5 erste positive Ergebnisse lieferte, zeigte er zugleich zentrale Schwächen auf: Die implementierte Lösung basiert auf einer statischen Integration des Datenmodells („Hardcodierung“). Daher erfordert jede Anpassung oder Erweiterung des Datenmodells manuelle Eingriffe in den Quellcode. Diese Tatsache schränkt Erweiterbarkeit, Wartbarkeit und Skalierbarkeit der Lösung erheblich ein. Die starre und unflexible Implementierung erweist sich langfristig als ineffizient und fehleranfällig.

Aus den dargelegten Gründen ergibt sich die Notwendigkeit einer dynamischen, generischen und datenmodellgesteuerten Lösung. Diese Anforderungen bilden die Grundlage für den Monidas Code Assist Navigator.

\newpage
\subsection{Monidas Code Assist Navigator} 

Info: Hier versuche ich das entwickelte Resultat in einem Szenario zu beschreiben. Dieser Abschnitt muss noch angepasst werden auf die vorherigen Kapitel. (TBD)

\paragraph{Szenario}
Stellen Sie sich folgendes Szenario vor: 

Sie sind Teil eines Entwicklerteams, das eine IoT-basierte Anwendung entwickelt. Aufgrund begrenzter zeitlicher und personeller Ressourcen ist es Ihrem Team jedoch nicht möglich, eine Benutzeroberfläche für die Konfiguration der Sensoren zu implementieren.

Um dennoch die ersten Sensoren zu verwalten, benötigen Sie eine Lösung, die ohne zusätzlichen Entwicklungsaufwand einsatzbereit ist. Genau hier setzt der im Rahmen dieser Arbeit entwickelte Monidas Code Assist Navigator an.

Zu Beginn definieren Sie innerhalb des Navigators das Schema Ihrer Anwendungsdomäne. Sobald das Datenmodell definiert und die Datenbank eingerichtet wurde, ist der Monidas Code Assist Navigator einsatzbereit.


%Um die Daetn zu verwalten muss eine Kompontente des Navigarotr insstaliert werden. ieren Sie zunächst eine dafür entwickelte VS-Code-Extension (Name noch festzulegen). Nach Sie alle definierten Sensoren über ein virtuelles Dateisystem verwalten. Das zugrunde liegende Datenmodell wird hierbei als hierarchischer Baum mit Ordnern und Dateien dargestellt. Wenn Sie eine dieser Dateien öffnen, erscheint automatisch der zugehörige Editor.

%An diesem Punkt unterstützt Sie die zweite Extension (Name noch festzulegen), welche als Editorunterstützung fungiert. Diese Extension erleichtert die Navigation innerhalb des virtuellen Dateisystems und bietet hilfreiche Funktionen wie automatisches Vervollständigen, Validieren von Eingaben und Anzeigen erforderlicher Attribute an. Dadurch wird die Verwaltung der IoT-Daten zusätzlich vereinfacht.


\subsection{Forschungsfragen}

Welche Architekturentscheidungen sind erforderlich, um den „Code Assist Navigator“ umzusetzen?

Welche Auswirkungen hat die Umsetzung des „Code Assist Navigators“ auf die Skalierbarkeit, Erweiterbarkeit und Flexibilität der bestehenden Monidas-Plattform, insbesondere im Hinblick auf das zugrundeliegende Datenmodell?

Wo liegen die technischen sowie konzeptionellen Grenzen des „Code Assist Navigators“ bei der Verarbeitung, Darstellung und Navigation komplexer und tiefer Datenmodelle?

\newpage
\subsection{Abgrenzung}
%Im Rahmen dieser Arbeit werden folgende Themenbereiche nicht behandelt. Die entwickelte Lösung wird nicht in das Produktivsystem von Monidas integriert. Es findet keine Anbindung an die bestehende Webplattform oder das Backend statt. Die Evaluation erfolgt ausschliesslich anhand des Datenmodells, ohne produktive Abläufe zu beeinflussen.

% Nicht berücksichtigt werden ausserdem Funktionen zur gleichzeitigen Nutzung durch mehrere Benutzer, die Benutzerverwaltung, Authentifizierungsprozesse sowie die Rechtevergabe. Auch Themen wie Datensicherheit, Performanceoptimierung, Datenmigration oder die Anbindung externer Systeme sind nicht Bestandteil dieser Arbeit.

Im Rahmen dieser Arbeit werden einige Themenbereiche bewusst ausgeklammert. Die entwickelte Lösung wird nicht in das Produktivsystem von Monidas integriert, und es erfolgt keine Anbindung an die bestehende Platform. Die Evaluation der Lösung basiert ausschliesslich auf dem Datenmodell.

Ebenfalls nicht berücksichtigt werden Funktionen zur Mehrbenutzernutzung, zur Benutzerverwaltung sowie zu Authentifizierungs- und Autorisierungsprozessen. Themen wie Datensicherheit, Performanceoptimierung oder Datenmigration liegen ebenfalls ausserhalb des Umfangs dieser Bachelorarbeit.

\subsection{Leserführung}

%Diese Bachelorarbeit ist in fünf Kapitel gegliedert. Kapitel 1 beschreibt die Ausgangslage und Problemstellung sowie einen Überblick zur entwickelten Lösung. Um den Lesenden einen verständlichen Einstieg in die Architektur zu ermöglichen, zeigt \textbf{Abbildung 1.1} eine vereinfachte Übersicht der entwickelten VS Code Extension. Diese Architektur besteht aus zwei zentralen Komponenten: der \textbf{Editorunterstützung (LSP-Komponenten)} und dem \textbf{Datenzugriff und der Verwaltung (VFS-Komponenten)}. Beide Komponenten kommunizieren über das Backend mit der Datenbank.


Info: Die Leseführung wird durch diese Darstellung unterstützt. Feedback wäre hilfreich, ob die aktuelle Darstellung zu detailliert ist und weiter reduziert werden sollte. 

\begin{figure}[H]
  \centering
  \includegraphics[width=\linewidth]{arch_simple.png}
  \caption{Architekturübersicht der implementierten Lösung}
  \label{fig:arch_simple}
\end{figure}


%Kapitel 2 widmet sich der Monidas-Plattform. Dort wird veranschaulicht, wie die Plattform aufgebaut ist und welche Technologien dabei verwendet werden. Zur besseren Verständlichkeit wird ein vereinfachtes Domänenmodell eingeführt und gezeigt, wie dieses in Form eines Schemas definiert wird. Dieses Schema bildet gleichzeitig die technische Grundlage der in der entwickelten VS Code Extension (Monidas Code Assist Navigator) verwendeten Technologien.

%Im Kapitel 3 steht das Schema der Datenbank im Mittelpunkt. Hier wird erläutert, weshalb das Schema für die Einführung des virtuellen Filesystems angepasst werden musste. Zusätzlich werden die selbst implementierten Constraints vorgestellt und deren Nutzen bei der Sicherstellung der Datenintegrität erläutert. Ein vereinfachtes Datenmodell aus dem Bereich E-Commerce dient als Referenzbeispiel für alle nachfolgenden Kapitel. Die Integration des ursprünglichen Monidas-Datenmodells erfolgt exemplarisch am Ende der Arbeit, um die Generalisierbarkeit der Lösung zu überprüfen. Abschliessend werden verschiedene Lösungsansätze präsentiert, wie das Schema zur Laufzeit analysiert und für die Komponenten VFS und LSP aufbereitet wird.

%Kapitel 4 beschäftigt sich mit dem virtuellen Filesystem. Im Vordergrund stehen dabei die VFS-Komponenten (4 und 3), welche eine dynamische Darstellung und Verwaltung der komplexen IoT-Daten ermöglichen.

%Kapitel 5 erläutert die Editorunterstützung durch die LSP-Komponenten (6, 7 und 8). Hier werden erweiterte Funktionen wie Validierung, Autovervollständigung und Navigation vorgestellt, welche eine effiziente und benutzerfreundliche Bearbeitung der IoT-Daten im Editor gewährleisten.

%Diese Struktur der Arbeit stellt sicher, dass die Lesenden eine klare Orientierung über die wesentlichen Inhalte erhalten und den Aufbau der entwickelten Lösung nachvollziehen können.

\section{Monidas-Plattform}
\label{mon}

Die Monidas-Plattform bildet die technische Grundlage für den Monidas Code Assist Navigator. Sie ist als IoT-Plattform konzipiert und ermöglicht die Erfassung, Verarbeitung und Überwachung von Sensordaten. Die technische Basis bilden dabei die Graphdatenbank \textit{Selva} sowie das Web-Framework \textit{Based}.

Dieses Kapitel beschreibt jene Aspekte der Monidas-Plattform, welche für die Umsetzung der in dieser Arbeit entwickelten Lösung relevant sind, um deren Kompatibilität mit der bestehenden Plattform sicherzustellen. Abschnitt~\ref{mon:plat} stellt hierzu zunächst die Entitäten vor, die für die Modellierung der Sensorüberwachung erforderlich sind. Abschnitt~\ref{abb:tech} erläutert anschliessend diejenigen Funktionalitäten und technologischen Eigenschaften von Selva und Based, auf denen die entwickelte Lösung aufbaut.

\subsection{Plattform}
\label{mon:plat}
Die Monidas-Plattform wurde in Kapitel \ref{kap:ein} im Zusammenhang mit dem Konfigurationsprozess eingeführt. Da dieser jedoch nur einen Teil der Gesamtfunktionalität abdeckt, werden in diesem Abschnitt Entitäten vorgestellt, die für die Überwachung von Sensoren relevant sind.

Zur Veranschaulichung zeigt Abbildung \ref{fig:mon_plat} die Dashboardansicht der Monidas-Plattform im Bereich „Trends“.  
Die Oberfläche listet mehrere Sensorgeräte (z.B. \textit{DatadotPilot01} bis \textit{DatadotPilot03}), die sich aktuell im Alarmzustand befinden. Für jedes Gerät werden Messdaten zur relativen Luftfeuchtigkeit und zur Temperatur der letzten 24 Stunden visualisiert – inklusive Minimal-, Maximal- und Mittelwert sowie grafischer Trendverläufe. Eine Filterleiste erlaubt die Einschränkung der Anzeige nach Gerätetyp und Status.

Die Benutzeroberfläche dient dabei lediglich der Veranschaulichung des Kontextes und ist nicht Gegenstand dieser Arbeit.  
Statt auf die grafische Oberfläche konzentriert sich diese Arbeit auf eine modellbasierte Verwaltungslogik, die direkt auf dem zugrunde liegenden Datenmodell aufbaut.  
Weitere UI-Ansichten der Monidas-Plattform werden daher im Folgenden nicht weiter berücksichtigt.

\begin{figure}[H]
  \centering
  \includegraphics[width=1\linewidth]{mon_plat.png}
 \caption{Dashboardansicht mit Trendverlauf von Sensoren im Alarmzustand}
  \label{fig:mon_plat}
\end{figure}

Die dargestellte Übersicht setzt voraus, dass zuvor bestimmte Entitäten im Datenmodell der Monidas-Plattform vorhanden sind. Aufgrund der Komplexität des vollständigen Domänenmodells beschränkt sich diese Darstellung auf einen relevanten Ausschnitt. Dieser Ausschnitt wird in Abbildung \ref{fig:mon_UML} anhand eines vereinfachten UML-Diagramms veranschaulicht.

In Kapitel~\ref{kap:eva} wird das dargestellte Modell erneut aufgegriffen und hinsichtlich der Auswirkungen der entwickelten Lösung auf das bestehende System evaluiert. Dabei stehen insbesondere die Forschungsfragen zur Skalierbarkeit, Erweiterbarkeit sowie zu den architektonischen Grenzen des zugrundeliegenden Datenmodells im Mittelpunkt der Analyse.

Die folgenden Entitäten bilden die Grundlage des betrachteten Modellausschnitts:
\begin{itemize}
  \item \textbf{User}: Personen, die einer Organisation oder einem Projekt zugeordnet sind.
  \item \textbf{Organization}: Organisationen.
  \item \textbf{Project}: Projekte, die einer Organisation zugeordnet sind.
  \item \textbf{Device}: Physische Geräte (z.B. Sensoren), die Daten erfassen und übertragen.
  \item \textbf{Monitor}: Überwachungseinheiten, die Sensordaten beobachten und Aktionen auslösen.
  \item \textbf{Function}: Wertet Sensordaten aus und ermittelt daraus den Zustand des Sensors.
  \item \textbf{Template}: Vorlage, welche den Betreff und Inhalt von Benachrichtigungen definiert.
  \item \textbf{Action}: Benachrichtigt definierte Empfänger bei bestimmten Sensorereignissen.
  \item \textbf{Group}: Gruppen, die festlegen, welche Personen Benachrichtigungen erhalten.
\end{itemize}

\begin{figure}[H]
  \centering
  \includegraphics[width=1\linewidth]{mon_UML.png}
  \caption{UML-Modellausschnitt der Monidas-Plattform zur Sensorüberwachung}
  \label{fig:mon_UML}
\end{figure}

Um das Zusammenspiel der oben beschriebenen Entitäten zu verdeutlichen, folgt ein praxisnahes Beispiel:  
An der FHNW (\textit{Organization}) gibt es das Projekt \textit{Serverraum Brugg} (\textit{Project}), in dem Techniker und Administratoren (\textit{User}) tätig sind. Im Serverraum sind mehrere Temperatursensoren (\textit{Devices}) installiert, die zur Überwachung der Raumtemperatur eingesetzt werden. Ein \textit{Monitor} prüft kontinuierlich die Temperatur und nutzt eine \textit{Function}, die in JavaScript implementiert ist, um zu erkennen, ob der Wert 27°C überschreitet.  
Wird dieser Grenzwert überschritten, ändert sich der Status auf \textit{Alarm}. Dies löst eine \textit{Action} aus, die eine Benachrichtigung an die zuständige \textit{Group} sendet. Der Inhalt der Benachrichtigung basiert auf einem \textit{Template} und lautet:

\begin{quote}
  „Die Temperatur im Serverraum Brugg hat 27°C überschritten. Bitte beobachten!“
\end{quote}


\subsection{Technologien}
\label{abb:tech}
Für die Umsetzung des virtuellen Dateisystems und des Language Servers werden Technologien verwendet, die gezielt auf die bestehende Infrastruktur der Monidas-Plattform abgestimmt sind. Diese Technologien wurden in Absprache mit dem Auftraggeber ausgewählt, um eine nahtlose Integration, Wartbarkeit und Erweiterbarkeit sicherzustellen. Im Folgenden werden nur jene Aspekte beschrieben, die für das Verständnis und die Umsetzung der in dieser Arbeit entwickelten Lösung relevant sind. Es ist anzumerken, dass die verfügbare Dokumentation der eingesetzten Technologien in Form von Repository-Dokumenten vorliegt, welche grundlegende Informationen bereitstellen und nur in begrenztem Umfang detaillierte Erläuterungen enthalten.

\subsubsection*{Selva}

Als Datenbanktechnologie wird die Graphdatenbank \textit{Selva}\footnote{\url{https://github.com/atelier-saulx/selva}} eingesetzt.

\paragraph{Datenbankschema}
Selva definiert das Datenmodell mithilfe eines Datenbankschemas, das Objekttypen, deren Felder sowie die jeweils erlaubten Datentypen beschreibt. Im Rahmen dieser Arbeit werden dabei nur jene Datentypen betrachtet, die in den in Abbildung~\ref{fig:mon_UML} dargestellten Entitäten tatsächlich verwendet werden.

Zu den verwendeten einfachen Datentypen zählen \texttt{string}, \texttt{boolean} und \texttt{number}, die beispielsweise zur Speicherung von Namen, Statuswerten oder numerischen Messwerten dienen. Validierte Datentypen wie \texttt{email} stellen sicher, dass nur syntaktisch korrekte E-Mail-Adressen gespeichert werden können. Strukturierte Daten, etwa Konfigurationen innerhalb von \textit{Function}-Entitäten, werden mithilfe des Datentyps \texttt{json} gespeichert. Für die Abbildung von Beziehungen zwischen Entitäten kommen die Referenztypen \texttt{reference} und \texttt{references} zum Einsatz, die den Kardinalitäten \texttt{0..1} bzw. \texttt{0..*} entsprechen. Selva erzwingt dabei jedoch keine wechselseitigen Einschränkungen wie etwa eine verpflichtende One-to-One-Beziehung. Solche komplexeren Anforderungen müssen bei Bedarf durch die Anwendung selbst implementiert werden.



Um die korrekte Funktion und Datenintegrität des Monidas Code Assist Navigators sicherzustellen, wurden ergänzende \textit{Constraints} definiert. Diese wurden direkt im Schema ergänzt. Da Selva jedoch nur die ihr bekannten Schema-Regeln auswertet, ignoriert sie diese Erweiterungen vollständig. Die neu definierten Constraints müssen daher von der Anwendung selbst geprüft und durchgesetzt werden. Alle für den Monidas Code Assist Navigator definierten Constraints sowie deren konkrete Umsetzung werden ausführlich in Kapitel~\ref{kap:dbschema} beschrieben.

Ein vereinfachtes Beispiel eines Selva-Schemas für die in Abbildung~\ref{fig:mon_UML} gezeigten Entitäten ist in Listing~\ref{lst:schema_beispiel} dargestellt. Es zeigt, wie die Entitäten \texttt{Organization}, \texttt{Project} und \texttt{User} mit ihren jeweiligen Feldern und Beziehungen definiert werden.

\newpage


\lstinputlisting[
  caption={Beispiel eines Selva-Datenbankschemas}, 
  label={lst:schema_beispiel},
  style=customtypescript
]{listings/schema_bsp.json}


\paragraph{Datenzugriff}
\label{sec:Datenzugriff}Um auf die im Schema definierten Objekte zugreifen und diese manipulieren zu können, stellt Selva eine API mit grundlegenden Operationen bereit. Mithilfe dieser API lassen sich Objekte erstellen und aktualisieren (\texttt{set}), abfragen (\texttt{get}), löschen (\texttt{delete}) sowie in Echtzeit überwachen (\texttt{observe}). Die konkrete Formulierung dieser Operationen erfolgt über eine eigene Query DSL, die auf JavaScript-Objekten basiert. Ähnlich wie beispielsweise bei GraphQL werden dabei die gewünschten Felder und Beziehungen präzise über eine strukturierte Objektdefinition angegeben.

In Selva existieren mehrere intern vordefinierte Felder, die nicht explizit im Schema definiert werden müssen. Im Rahmen dieser Arbeit sind die Felder \textit{ID} und \textit{Aliases} relevant. Die ID dient der eindeutigen Identifikation eines jeden Objekts und wird entweder durch Selva generiert oder manuell vorgegeben. Wird keine ID angegeben, erzeugt die Datenbank eine neue ID. In diesem Fall muss zwingend der Typ des Objekts (z.B. \texttt{Organization}) angegeben werden. Zudem bietet Selva die Möglichkeit, deterministisch immer dieselbe ID zu erzeugen. Zusätzlich zur ID besitzen alle Objekte Aliases. Ein Objekt kann mehrere Aliases besitzen, und es ist möglich, dass verschiedene Objekte denselben Alias verwenden. Ihre genaue Bedeutung und Verwendung hängt von der jeweiligen Anwendungslogik ab. Grundsätzlich dienen Aliases dazu, Datenobjekte benutzerfreundlicher und flexibler abzurufen oder zu referenzieren.

Listing~\ref{lst:selva_bsp} veranschaulicht die beschriebenen Operationen in einem praxisnahen Anwendungsfall. Zunächst wird eine neue Organisation mit einem Alias erstellt. Danach wird diese sowohl über die automatisch generierte ID als auch über den definierten Alias abgefragt. Anschliessend demonstriert das Beispiel die Überwachung von Echtzeitänderungen am Objekt. Abschliessend wird gezeigt, wie mithilfe eines externen Alias eine eindeutige ID generiert werden kann, um gezielt ein weiteres Objekt zu erstellen und dieses anschliessend zu löschen. Die in diesem Beispiel dargestellten Operationen bilden die Basis für die entwickelte Lösung, insbesondere für die Umsetzung des virtuellen Filesystems und des Language Servers.

\newpage


\lstinputlisting[
  caption={Beispiel zur Verwendung der Selva-API}, 
  label={lst:selva_bsp},
  style=customtypescript
]{listings/selva_bsp.js}

\subsubsection*{Based}
PLS HELP Muesii das überhaupt ihfüre wenni es eig nie erwähnt ?

Selva kanni jo lo welli aliases usw verwänd unds Schema oder ?
%\section{Datenmodell und Schema}
Leseeinführung 

\subsection{Struktur}
Aufbau der Typenstruktur, Felder, Referenzen.

\subsection{Constraints}

\begin{itemize}
	\item Definition von constraints wie notNull, existsIn
	\item Nutzung zur Validierung im VFS und zur Vervollständigung im LSP
\end{itemize}

\subsection{Beispielmodell}
Einheitliches Modell für alle Kapitel

\subsection{Analyse des Schemas}
Traversieren des Schemas: - Erkennen von Wurzeln (Root-Typen) - Analyse von Referenzpfa 
den - Erkennung von moeglichen Zyklen


%\section{Architektur}
tbd

\subsection{Architekturübersicht}
Alter Stand tbd
Nach der Vorstellung der verwendeten Technologien beschreibt dieser Abschnitt den Aufbau der Gesamtarchitektur. Für das Lesen der Arbeit stellt die Übersicht eine Orientierungshilfe dar, um die technische Gliederung der Lösung von Beginn an nachvollziehbar zu machen.

Die Lösung ist modular aufgebaut und besteht aus zwei getrennten Extensions. Eine ist für das virtuelle Filesystem zuständig, die andere für die Editorunterstützung. Der Editor ist die Schnittstelle, über die die Editorunterstützung auf die Inhalte des virtuellen Filesystem zugreift.

...tbd


\begin{figure}[H]
  \centering
  \includegraphics[width=\linewidth]{Arch.png}
  \caption{Architekturübersicht der implementierten Lösung}
  \label{fig:arch_modell}
\end{figure}

\section{Schema}



\subsection{Domänenmodell}
Zur einheitlichen Darstellung und besseren Verständlichkeit der entwickelten Lösung wurde für diese Arbeit ein Domänenmodell aus dem Bereich E-Commerce definiert. Dieses Modell bildet ein beispielhaftes Szenario ab, das für die nachfolgenden Kapitel als Referenz dient. Die entwickelte Lösung ist jedoch nicht auf dieses spezielle Modell beschränkt, sondern kann auf beliebige andere Domänenmodelle übertragen werden, ohne dass Änderungen an der Kernfunktionalität nötig sind. Im diesem Abschnitt wird das erstellte Domänenmodell basierend auf der UML-Darstellung in Abbildung~\ref{fig:uml_modell} erläutert.

Info:
Das hier gezeigte Modell wird noch detaillierter beschrieben. Das Feedback von Herrn Gruntz bezüglich der UML-Darstellung wurde bisher noch nicht eingearbeitet.

\begin{figure}[H]
  \centering
  \includegraphics[width=\linewidth]{UML.png}
  \caption{UML-Darstellung des für diese Arbeit definierten Domänenmodells.}
  \label{fig:uml_modell}
\end{figure}

Das definierte Domänenmodell besteht aus den Entitäten \texttt{Customer}, \texttt{Order}, \texttt{Address}, \texttt{Product} und \texttt{LineItem}. Zur eindeutigen Identifikation im virtuellen Filesystem verfügt jede Entität zwingend über das Attribut \texttt{slug}. Der \texttt{slug} ist ein menschenlesbarer Bezeichner, der zur Generierung eindeutiger URI-Pfade verwendet wird. Wird beispielsweise ein Kunde mit dem \texttt{slug} „max“ erstellt, entsteht daraus der eindeutige URI-Pfad \texttt{/Customers/max}. Die konkrete Bedeutung und Nutzung des Attributs \texttt{slug} im Kontext des virtuellen Filesystems wird in Abschnitt~\ref{vfs-umsetzung} detailliert beschrieben.
\section{Virtuelles Filesystem}

\subsection{Ziel und Überblick}
Dieses Kapitel beschreibt die Ziele, den Aufbau und die Funktionen des virtuellen Dateisystems (VFS), das zur Navigation und Bearbeitung von Instanzen innerhalb der graphbasierten Datenbank dient.

\subsection{Explorer-Baum}
Der Explorer-Baum stellt die logische Struktur der Instanzen hierarchisch dar. Basierend auf dem Schemamodell werden gültige Pfade generiert und zyklische Referenzen erkannt und behandelt. Ziel ist eine konsistente, intuitive Navigation vergleichbar mit einem klassischen Dateisystem.

\subsubsection{Problemstellung}
Die Herausforderung besteht darin, komplexe Referenzbeziehungen und potenzielle Zyklen so zu verarbeiten, dass eine eindeutige, strukturierte Baumansicht entsteht.

\subsubsection{Lösungsansätze}
Zur Lösung wird eine Tiefensuche mit Zykluserkennung eingesetzt. Dabei werden Root-Typen ermittelt und gültige Pfadsegmente anhand der `UriTree`-Struktur berechnet.

\subsubsection{Umsetzung}
\label{vfs-umsetzung}
Die Baumstruktur wird dynamisch zur Laufzeit generiert. Zyklische Verweise werden durch Alias-Referenzen dargestellt, um Redundanzen und Endlosschleifen zu vermeiden.

\subsection{Editor}
Der Editor zeigt den Inhalt einer Instanz als strukturierte JSON-Datei an. Änderungen erfolgen direkt im Dateitext und werden über folgende Funktionen verarbeitet:
\begin{itemize}
	\item \texttt{createDir} – Erzeugt eine neue Instanz inklusive Pfad und Meta-Informationen
	\item \texttt{writeFile} – Aktualisiert den Inhalt einer bestehenden Instanz
\end{itemize}

\subsubsection{Problemstellung}
Es muss sichergestellt werden, dass beim Erstellen oder Bearbeiten einer Datei sowohl die Pfadstruktur als auch die Inhaltsvalidität berücksichtigt werden.

\subsubsection{Lösungsansätze}
Zur Unterstützung der Benutzerinteraktion werden Templates generiert und dynamisch ergänzt, je nachdem, ob Constraints definiert sind oder nicht.

\subsubsection{Umsetzung}
Die Darstellung unterscheidet zwei Modi:
\begin{itemize}
	\item \textbf{Ohne Constraints:} Freie Eingabe aller Felder ohne Validierungsvorgaben
	\item \textbf{Mit Constraints:} Automatisch erzeugte Eingabehilfen und Validierungsvorgaben basierend auf dem Schemamodell
\end{itemize}

\subsection{Validierung}
Die Validierung erfolgt durch eine Kombination aus statischer Zod-Validierung und der Prüfung schemabedingter Constraints. Dabei wird sichergestellt, dass alle Pflichtfelder gesetzt und alle referenzierten Objekte gültig sind.

\subsection{Notifikationen}
Benutzer erhalten Rückmeldung über Systemaktionen sowohl im Explorer als auch im Editor. Beispiele:
\begin{itemize}
	\item \textbf{Explorer:} Rückmeldungen bei Erstellen und Löschen von Instanzen
	\item \textbf{Editor:} Hinweise bei Validierungsergebnissen oder beim Speichern von Änderungen
\end{itemize}

\section{Editorunterstützung }
\label{lsp}
\subsection{Ziel und Überblick}
Was soll die LSP-Integration leisten? (Navigation, Validierung, Vervollständigung) etc.

\subsection{Problemstellung}
Welche Ansätze gibt es zur Unterstützung im Editor?

\subsection{Umsetzung}

\subsubsection{Validierung}
Welche Felder prüft der LSP? Mit und ohne Constraints. 
Live-Validierung beim Schreiben im Editor.

\subsubsection{Autovervollständigung}
Welche Daten werden angezeigt? Wie werden die Daten geholt – mit und ohne Constraints.

\subsubsection{Go-to-Definition}
Wie funktioniert Go-to-Definition? Momentan über die ID, die im JSON gesetzt ist, usw.
\section{Schlussbemerkungen}
\label{kap:eva}

\subsection{Beantwortung der Fragestellungen}

\subsection{Kritische Würdigung}

\subsection{Ausblick und zukünftige Erweiterung}


%%---BIBLIOGRAPHY------------------------------------------------------------------------
{\sloppypar
\printbibliography[heading=bibintoc, title=Quellenverzeichnis]
}

%%---APPENDIX----------------------------------------------------------------------------
\section*{Eigenständigkeitserklärung}
\markboth{\MakeUppercase{Eigenständigkeitserklärung}}{\MakeUppercase{Eigenständigkeitserklärung}}

\addcontentsline{toc}{section}{Eigenständigkeitserklärung}

Ich erkläre hiermit, dass ich den vorliegenden Leistungsnachweis selber und selbständig verfasst habe,
\begin{itemize} 
\item dass ich sämtliche nicht von mir selber stammenden Textstellen und anderen Quellen wie Bilder etc. gemäss gängigen wissenschaftlichen Zitierregeln\footnote{IEEE} korrekt zitiert und die verwendeten Quellen klar sichtbar ausgewiesen habe; 
\item dass ich in einer Fussnote oder einem Hilfsmittelverzeichnis alle verwendeten Hilfsmittel (KI-Assistenzsysteme wie Chatbots\footnote{z.B. ChatGPT}, Übersetzungs-\footnote{z.B. Deepl} Paraphrasier-\footnote{z.B. Quillbot} oder Programmierapplikationen\footnote{z.B. Github Copilot}) deklariert und ihre Verwendung bei den entsprechenden Textstellen angegeben habe;
\item dass ich sämtliche immateriellen Rechte an von mir allfällig verwendeten Materialien wie Bilder oder Grafiken erworben habe oder dass diese Materialien von mir selbst erstellt wurde;
\item dass das Thema, die Arbeit oder Teile davon nicht bei einem Leistungsnachweis eines anderen Moduls verwendet wurden, sofern dies nicht ausdrücklich mit der Dozentin oder dem Dozenten im Voraus vereinbart wurde und in der Arbeit ausgewiesen wird; 
\item dass ich mir bewusst bin, dass meine Arbeit auf Plagiate und auf Drittautorschaft menschlichen oder technischen Ursprungs (Künstliche Intelligenz) überprüft werden kann;
\item dass ich mir bewusst bin, dass die Hochschule für Technik FHNW einen Verstoss gegen diese Eigenständigkeitserklärung bzw. die ihr zugrundeliegenden Studierendenpflichten der Studien- und Prüfungsordnung der Hochschule für Technik verfolgt und dass daraus disziplinarische Folgen (Verweis oder Ausschluss aus dem Studiengang) resultieren können.
\end{itemize}

\vspace*{4ex}

Windisch, 14. August 2025

\vspace*{4ex}

{\renewcommand{\arraystretch}{1.5}
\begin{tabular}{@{}>{\bf}ll}
Name: & Gianni Parrillo\\
Unterschrift: & \\
\end{tabular}
\selectlanguage{ngerman}				%ngerman or english
\begin{appendix} % Anhang
\section{Ein Anhang}

\subsection{Aufgabenstellung im Originalwortlaut}

\subsection{Gesamtübersicht}

\subsection{Berechnungen / Resultate Umfrage}

\subsection{Tests – Screenshots}

\subsection{Code-Beispiele} 


\lstinputlisting[
  caption={}, 
  label={lst:extended-schema},
  style=customtypescript
]{listings/extended-schema.ts}



%%---NOTES for DEBUG---------------------------------------------------------------------
%\newpage
%\listoftodos[\section{Todo-Notes}]
%\clearpage

\end{document}