\section{Einleitung}

\subsection{Ausgangslage und Problemstellung}
Die modulare Entwicklung von Software wird im Bereich der IoT-Systeme zunehmend relevant. Mit der steigenden Anzahl vernetzter Geräte und wachsenden Datenmengen nimmt die Komplexität zu. Modulare Lösungen gelten als zentraler Bestandteil aktueller IoT Ansätze \cite{IoT}.

Die Colomba Link GmbH betreibt mit Monidas eine IoT-Plattform für industrielle Anwendungen. Sie verwaltet Organisationen, Projekte, Benutzer, Sensoren sowie zugehörige Überwachungs- und Benachrichtigungsregeln. Mit zunehmender Anzahl an Sensoren und Konfigurationen steigt der Aufwand für technische Mitarbeitende. Sie sind für die Einrichtung und Konfiguration der Sensoren zuständig. Verknüpfungen zwischen Sensoren, Regeln und Benachrichtigungen müssen in mehreren Schritten bearbeitet werden.

Um diesen Prozess zu optimieren, wurde im Rahmen eines Vorprojekts ein Proof of Concept erstellt. Ziel war es, die Machbarkeit einer alternativen Schnittstelle in Form einer VS Code Extension zu evaluieren. Dafür wurde ein Prototyp entwickelt, der die Plattformdaten in einer TreeView darstellt und deren Bearbeitung über einen JSON-Editor ermöglicht.
 
Der Proof of Concept erfüllte die funktionalen Anforderungen. Das Domänenmodell war jedoch fest im Code definiert. Änderungen am Datenmodell erforderten manuelle Anpassungen in der Extension. Diese Einschränkung bildet die Grundlage für die vorliegende Arbeit.

\subsection{Zielsetzung}



\subsection{Fragestellung}
Das vorliegende Problem mit den oben beschriebenen Einschränkungen führt zu folgenden Fragestellungen:

\begin{itemize}
  \item Welche Vor- und Nachteile ergeben sich beim Wechsel von einer statisch implementierten VS Code Extension zu einem generischen, schemabasierten Datennavigator mit Dateisystemstruktur? 
  
\vspace{0.5em}

  \item Welche Auswirkungen hätte dieser Architekturwechsel auf den im Stand IP5 implementierten Custom TreeView, und welche Vor- und Nachteile hätte ein Wechsel zur Filesystem View?

    \vspace{0.5em}
  
  \item Wie kann ein eigener JSON-LSP-Server entwickelt werden, um eine Navigation innerhalb des Datenbank-Navigators zu ermöglichen?
\end{itemize}



\subsection{Leserführung}
Was wird wo behandelt?