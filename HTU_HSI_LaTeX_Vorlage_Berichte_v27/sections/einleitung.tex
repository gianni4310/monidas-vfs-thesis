\section{Einleitung}
In diesem Kapitel wird die Auftraggeberfirma Colomba Link vorgestellt. Es werden die Ausgangslage beschrieben, das zu lösende Problem eingeführt und grundlegende Begriffe erläutert, die für das Verständnis dieser Thesis von Bedeutung sind.

\subsection{Colomba Link}
Die Colomba Link GmbH entwickelt IoT-Lösungen für industrielle Anwendungen. Eines ihrer zentralen Produkte ist Monidas, ein System zur Erfassung, Übertragung und Verwaltung von Sensordaten. Es besteht aus einem Sensorgerät und einer Plattform. Das Gerät kann je nach Ausführung verschiedene Werte wie Temperatur, Luftfeuchtigkeit, Energieverbrauch, Luftqualität, Präsenz, Position oder Vibration messen. Die erfassten Daten werden per LoRaWAN an die Plattform übermittelt. Über ein Web-Dashboard können Nutzer Geräte verwalten, Alarme definieren und Konfigurationen anpassen.

Mit wachsender Anzahl an Geräten und Projekten wurde die Verwaltung komplexer. Um die Konfiguration gezielter bearbeiten zu können, wurde im Vorprojekt IP5 eine VS Code Extension als alternative Schnittstelle entwickelt. Sie visualisiert die Datenstruktur in einer Baumansicht und ermöglicht die direkte Bearbeitung über einen integrierten Editor.

\subsection{Monidas Superuser}


\subsection{Problemstellung}
Beschreibung der Problemstellung

\subsection{Fragestellung}


\subsection{Resultat}
Übersicht über die umgesetzte Lösung

\subsection{Systemübersicht}

\subsection{Systemübersicht}

\textbf{Graphdatenbank}
\begin{itemize}
	\item db.schema …
\end{itemize}

\textbf{Virtuelles Filesystem}
\begin{itemize}
	\item vfs-vscode-extension für VFS
	\item based-vfs
\end{itemize}

\textbf{Language Server Protocol}
\begin{itemize}
	\item lsp-vscode-extension für LSP
	\item based-lsp
\end{itemize}

\subsection{Leserführung}
Was wird wo behandelt?