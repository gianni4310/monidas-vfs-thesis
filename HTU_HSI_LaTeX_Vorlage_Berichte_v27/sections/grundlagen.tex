\section{Datenmodell und Schema}

\subsection{Struktur}
Dieses Kapitel beschreibt den Aufbau der Typenstruktur innerhalb des Schemas. Es werden insbesondere die Felddefinitionen, Referenzverknüpfungen sowie die innere Struktur einzelner Knoten erläutert.

\subsection{Constraints}
Zur Validierung und Strukturierung des Datenmodells werden unterschiedliche Constraints eingesetzt. Dazu zählen beispielsweise:
\begin{itemize}
	\item \texttt{notNull} – zur Sicherstellung, dass bestimmte Felder zwingend ausgefüllt sind
	\item \texttt{existsIn} – zur Durchsetzung, dass referenzierte Objekte existieren müssen
\end{itemize}
Diese Constraints werden sowohl im virtuellen Dateisystem (VFS) zur Validierung beim Schreiben als auch im Language Server (LSP) für Vorschläge und automatische Vervollständigungen verwendet.

\subsection{Beispielmodell}
Für die nachfolgenden Kapitel wird ein einheitliches Beispielmodell verwendet. Es illustriert typische Strukturen und Referenzbeziehungen und dient als durchgängige Grundlage zur Veranschaulichung.

\subsection{Analyse des Schemas}
Zur Analyse des Schemas wird eine Traversierung über die Typstruktur durchgeführt. Dabei werden insbesondere folgende Aspekte betrachtet:
\begin{itemize}
	\item Identifikation von Root-Typen (Wurzeln der Baumstruktur)
	\item Ableitung gültiger Referenzpfade auf Basis der Feldverknüpfungen
	\item Erkennung potenzieller Zyklen innerhalb des Schemas
\end{itemize}
Diese Analyse bildet die Basis für die Pfadgenerierung im virtuellen Dateisystem.
